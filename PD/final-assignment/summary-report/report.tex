\documentclass[11pt]{article}
\usepackage{graphicx}
\usepackage[margin=2.5cm]{geometry}
\usepackage{tikz}
\usepackage{indentfirst}
\usepackage{tabularx}
\usepackage{listingsutf8}
\usepackage{color}
\usepackage{hyperref}
\usepackage{float}
\usepackage[portuguese]{babel}

\graphicspath{{./images/}}

\def\checkmark{\tikz\fill[scale=0.4](0,.35) -- (.25,0) -- (1,.7) -- (.25,.15) -- cycle;} 
\setlength{\parskip}{0.5em}

\renewcommand{\lstlistingname}{Código}
\renewcommand{\lstlistlistingname}{Pedaços de Código}

\definecolor{dkgreen}{rgb}{0,0.6,0}
\definecolor{gray}{rgb}{0.5,0.5,0.5}
\definecolor{mauve}{rgb}{0.58,0,0.82}

\hypersetup{
	colorlinks=false,
	linktoc=all,
	hidelinks,
}

\lstset{
	belowcaptionskip=1\baselineskip,
	captionpos=b,
	frame=tb,
	language=Java,
	aboveskip=3mm,
	belowskip=3mm,
	showstringspaces=false,
	columns=flexible,
	basicstyle={\small\ttfamily},
	numbers=none,
	numberstyle=\tiny\color{gray},
	keywordstyle=\color{blue},
	commentstyle=\color{dkgreen},
	stringstyle=\color{mauve},
	breaklines=true,
	breakatwhitespace=true,
	tabsize=3,
	inputencoding=utf8,
	extendedchars=true,
	literate={á}{{\'a}}1 {ã}{{\~a}}1 {à}{{\`a} }1 {Ã}{{\~A}}1 {ó}{{\'o}}1 {Ó}{{\'O}}1 {Í}{{\'I}}1 {í}{{\'i}}1 {é}{{\'e}}1 {ç}{{\c{c}}}1 {Ç}{{\c{C}}}1 {ú}{{\'u}}1 {õ}{{\~o}}1
}

\begin{document}
	\begin{titlepage}
		\begin{center}
			\includegraphics[width=0.6\textwidth]{logo-isec}
			
			\vspace*{\fill}
			
			\includegraphics[width=0.4\textwidth]{icon-metapd}
			
			\Huge
			\textbf{MetaPD}
			
			\huge
			Relatório Sumário - Meta 3
			
			\vspace{2cm}
			
			\Large
			\textbf{
				Ângelo Paiva, 2019129023 \\
				Daniel Ribeiro, 2017013425 \\
				Francisco Ferreira, 2015017086
			}
			
			\vfill
			\vspace*{\fill}
			
			\normalsize
			Programação Distribuída \\
			Licenciatura de Engenharia Informática, Ramo de Desenvolvimento de Aplicações \\
			22 de janeiro de 2022		
		\end{center}
	\end{titlepage}

	\tableofcontents
	\pagebreak
	
	\large
	\section{RESTAPI}
	\normalsize
	
	A RESTAPI foi escrita em SpringBoot. Para todos os pedidos, menos para pedir token, é necessário autenticação (sendo que cada token expira passado 2 minutos).
	
	Nesta fazemos uso de \textbf{PUT}, \textbf{POST}, \textbf{GET} e \textbf{DELETE}.
		
	A especificação desta pode ser encontrada no ficheiro \textbf{rest-api-specification.pdf} que foi entregue junto deste relatório. Esta foi escrita em markdown e compilada para PDF usando \color{blue}\underline{\href{https://md2pdf.netlify.app/}{este website}}\color{black}.
	
	
	\large
	\section{Interfaces Remotas (Java RMI)}
	\normalsize
	
	Foram desenvolvidas duas interfaces remotas, \textbf{RemoteLbObserverInterface} e \textbf{RemoteLbObservableInterface}. Esta última interface é a que fica associada ao load-balancer.
	
	O observer, ao iniciar, cria uma instância do objeto remoto, procura o observable (que, no caso, é o load-balancer) registado com o nome de serviço \textbf{GRDS\_Service} e, quando obtém uma referência para este, regista-se como observador.
	
	O observable, sempre que assim é pretendido, envia uma notificação aos observers a dizer que algo aconteceu. Por exemplo, no caso duma notificação de alteração da base de dados, os observers recebem uma notificação a dizer que houve uma alteração à base de dados em conjunto com uma lista dos utilizadores afetados.
	
	
\end{document}